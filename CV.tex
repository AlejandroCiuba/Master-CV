\documentclass[11pt]{tianyicv}

\usepackage[sfdefault, mono=false]{libertinus}
\usepackage[sansmath]{libertinust1math}

% \hypersetup{nolinks=true}           % Disable hyperlinks
% \geometry{showframe}                % Show page frames for debugging purposes
\geometry{margin=0.75in}

\NewDocumentCommand{\dates}{mm}{#1 -- #2}

\bibliography{citations/pubs}

\title{Curriculum Vitae}
\author{Alejandro Ciuba}
\email{alejandrociuba@pitt.edu}
\phone{(412) 874-1838}
\homepage{alejandrociuba.github.io}
\address{%
    Department of Computer Science\\
    University of Pittsburgh\\
    135 N Bellefield Ave\\
    Pittsburgh, PA 15213
}
\github{github.com/AlejandroCiuba}
% \linkedin{https://www.linkedin.com/in/alejandro-ciuba/}
% \date{Current Date}


\begin{document}

\maketitle

\section{Education}

\phd{University of Pittsburgh}
{Ph.D. in Computer Science}
{Pittsburgh, PA}{\dates{August 2023}{Present}}
{Community-focused natural-language processing}{
    \begin{itemize}
        \item \textbf{Advisors:} Aakash Gautam and Lorraine (Xiang) Li 
    \end{itemize}
}

\education{University of Pittsburgh}
{B.S. in Computer Science}
{Pittsburgh, PA}{\dates{August 2019}{May 2023}}
{Minors in Linguistics and Spanish}{}[3.97]


\section{Research Experience}

\experience{Human-Computer Interaction Research: Low-Resource Language Preservation}
{Graduate Student Researcher under Dr. Aakash Gautam}
{Pittsburgh, PA}{\dates{January 2024}{Present}}{
    \begin{itemize}
        \item Working on a literature review of NLP community engagement practices using the PRISMA framework
        \item Conducting interviews and thematic analysis to better understand how technology can benefit communities
    \end{itemize}
}

\experience{Natural Language Processing Research: Automatic Essay Evaluation}
{Graduate Student Researcher under Dr. Lorraine (Xiang) Li}
{Pittsburgh, PA}{\dates{May 2024}{Present}} {
    \begin{itemize}
        \item Performing baseline training/fine-tuning on a custom BERT-based model via PyTorch and HuggingFace 
        \item Conducting exploratory data analysis on essay corpora via feature correlation and hypothesis testing
    \end{itemize}
}

\experience{ASR Transcription Research \& Technical Support}
{Research Assistant under Dr. Dan Villarreal}
{Pittsburgh, PA}{\dates{August 2022}{August 2024}}{
    \begin{itemize}
        \item Developed custom ASR pipeline with PyTorch Lightning and \texttt{pyannote} to automatically segment audio
        \item Developed a Python API to work with ELAN files with \texttt{pytest} integration
    \end{itemize}
}

\experience{Classroom Discussion Analysis Research}
{Research Assistant under Dr. Diane Litman}
{Pittsburgh, PA}{\dates{January 2023}{April 2023}}{
    \begin{itemize}
        \item Managed dataset cleaning pipeline for Word Documents, focusing on anonymization and XML tag extraction
        \item Examined teacher-student classroom discussion differences through data analysis and hypothesis testing 
    \end{itemize}
}

\experience{Frederick Jelinek Memorial Summer Workshop}
{Undergraduate Researcher under Drs. Anthony Larcher and Santosh Kesiraju}
{Baltimore, MD}{\dates{June 2022}{August 2022}}{
    \begin{itemize}
        \item Created a trilingual corpus (English, French and Tamasheq/Tamahaq) for low-resource machine translation
        \item Ran low-resource model training simulations (e.g. for T5-Small and mBART)
    \end{itemize}
}

\experience{Linguistics Research}
{Undergraduate Research Assistant under Drs. Alan Juffs and Na-Rae Han}
{Pittsburgh, PA}{\dates{January 2022}{April 2022}}{
    \begin{itemize}
        \item Parsed video game scripts for pragmatics analysis using \texttt{spaCy}, \texttt{nltk} and \texttt{beautifulsoup}
        \item Aided in the cleaning of the unpublished PELIC Speech Dataset
    \end{itemize}
}

\section{Work Experience}

\experience{Teaching Assistant Positions}
{Computational Linguistics and Data Structures \& Algorithms}
{Pittsburgh, PA}{\dates{August 2021}{April 2022}}{
    \begin{itemize}
        \item Ran recitations/office hours for students, covering fundamental subject concepts
        \item Ran a class lecture on machine translation with Jupyter Notebook demo
    \end{itemize}
}

\experience{Carnegie Library ESL Class}
{Volunteer ESL Instructor for the Carnegie Public Library}
{Pittsburgh, PA}{\dates{August 2019}{March 2020}}{
    \begin{itemize}
        \item Guided adult learners through vocabulary and grammar exercises
        \item Maintained a safe environment for English learners
    \end{itemize}
}

\section{Publications}
% Entry command is basically the same as the Experience command, but I kept them separate so that I may choose to style them differently

% \publication{INTERSPEECH}
% {Strategies for improving low resource speech to text translation relying on pre-trained ASR models}{May 2023}
% \cite{kesirajuStrategiesImprovingLow2023}

% \publication{HAL}
% {Multi-lingual Speech to Speech Translation for Under-Resourced Languages}{August 2022}

\begin{publications}[INTERSPEECH]
    \publication{\cite{kesirajuStrategiesImprovingLow2023}}
\end{publications}

\begin{publications}[Horizons 2020 Project]
    \publication{\cite{larcherMultilingualSpeechSpeech2022_mod}}
\end{publications}

\section{Awards}

\award{Southern Regional Educational Board Member}{\dates{August 2024}{Present}}

\award{K. Leroy Irvis Fellow}{\dates{August 2023}{Present}}

\award{CSC Hacks Winner + Best Beginner Project}{November 2020}[\normalfont Submission: \textit{Flatland: The Game}]

\award{Provost Academy Member}{\dates{August 2019}{May 2023}}

\section{Skills}
% Entry command is basically the same as the Experience command, but I kept them separate so that I may choose to style them differently

\skill{Programming \& Computation}[
    \begin{itemize}
        \item C, Python, R, Go, Java
        \item Git, GitHub and GitLab
        \item Compute cluster usage via Slurm
        \item Linux, Docker and \texttt{bash} script creation
    \end{itemize}
]

\skill{User Research Skills}[
    \begin{itemize}
        \item Quantitative analysis via inferential statistics
        \item Qualitative analysis via coding and thematic analysis
        \item User interview and survey creation
    \end{itemize}
]

\skill{Foreign Languages \& Linguistics}[
    \begin{itemize}
        \item Conversational Spanish
        \item Basic Portuguese Comprehension
        \item Manual document translation experience
    \end{itemize}
]

\end{document}